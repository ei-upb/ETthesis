% Template for typesetting a document using the "etthesis" class.
% See README.md for further information.

\documentclass[
    % Note: See README.md for details on class options.
]{etthesis}

% Custom packages and configuration are handled separately.
\usepackage{custom}

% Title information
\title{Hier steht der meist umfang- und fachterminusreiche Titel der jeweiligen akademischen Arbeit}
\author{Max Mustermann} % Format: <First> <Middle> <Last>
\date{November 2018} % Format: <Month> <Year>, e.g. "November 2018".
\group{Fachgebiet Elektrische Messtechnik} % Group the document is written at.
\grouplogo{} % Path to logo of the group the document is written at (optional).
\keywords{Fachterminus, Schlüsselwort} % Keywords describing the document. Required of PDF/A-2b compliance.

% Advisor names
% For bachelor's and master's theses, advisor A is the grading professor and
% advisor B is the advising staff member. For doctoral theses, these are the
% two official reviewers.
\advisorA{Prof.~Dr.-Ing.~Bernd Henning}
\advisorB{Erika Mustermann, M.Sc.}

% Additional information for student theses
\salutation{Herrn} % "Frau" or "Herrn" (do not use academic degree).

% Additional information for doctoral theses
% \nameprefix{Dipl.-Ing.} % Academic degree set before the name.
\namesuffix{M.Sc.} % Academic degree set after the name.
\birthdate{01.01.1970} % Author's birth date, given as DD.MM.YYYY. Used on the submission title page.
\birthplace{Paderborn} % Author's place of birth. Used on the submission title page.
\examdate{15.08.2018} % Day of the oral exam, given as as DD.MM.YYYY. Used on the accepted title page.
\thesisnumber{123} % Thesis number ("EIM-E/" prefix is added automatically). Used on the accepted title page.

\begin{document}
    % The entire frontmatter is automatically generated by the etthesis class.
    % Depending on the selected document type, content is automatically
    % included and can be customized in the following files in the frontmatter
    % subdirectory (see README.md for further details):
    % * preface.tex
    % * problemstatement.tex
    % * abstract_de.tex
    % * abstract_en.tex
    % * symbols.tex

    % \input your regular content chapters (mainmatter).
    \chapter{Einleitung}
Hier erscheint die Einleitung.

    % TODO: Add your own chapters here.
    \chapter{Zusammenfassung und Ausblick}
Hier erscheinen Zusammenfassung und Ausblick.


    % After the regular content, the appendix follows, starting with the bibliography.
    \appendix

    % Insert the bibliography, using the correct single line spacing page layout.
    % Note: When creating a doctoral thesis in submission variant (i.e.
    % doctstate=submission), the main bibliography does not include the
    % author's own publications. These are put into a separate list. See below
    % for details.
    \insertbibliography%

    % TODO: Add your own appendices here.
    \chapter{Mathematischer Anhang}
Hier erscheint ein mathematischer Anhang.


    % A backmatter is automatically created depending on the selected document
    % type and can be customized in the following files in the backmatter
    % subdirectory (see README.md for further details):
    % * cv.tex
    % * affirmation.tex
\end{document}
