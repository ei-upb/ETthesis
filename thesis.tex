\documentclass[
]{emtthesis}

% Always use UTF-8 for input encoding.
% Don't forget to configure your editor accordingly!
\usepackage[utf8]{inputenc}

% Custom packages and configuration are handled separately.
\usepackage{custom}

% Title information
\title{Hier steht der meist umfang- und fachterminusreiche Titel der jeweiligen akademischen Arbeit}
\author{Max Mustermann} % Format: <First> <Middle> <Last>
\date{November 2018} % Format: <Month> <Year>, e.g. "November 2018".

% Additional author information
\salutation{Herrn} % "Frau" or "Herrn".

% Advisor names
% For B.Sc. and M.Sc. theses, advisor A is the grading professor and advisor B is the advising staff member.
% For doctoral theses, these are the two official reviewers.
\advisorA{Prof.~Dr.-Ing.~Bernd Henning}
\advisorB{Erika Mustermann, M.Sc.}

% Additional information for doctoral theses
% \nameprefix{Dipl.-Ing.} % Academic degree set before the name.
\namesuffix{M.Sc.} % Academic degree set after the name.
\birthdate{01.01.1970} % Author's birth date, given as DD.MM.YYYY. Used on the unaccepted title page.
\birthplace{Paderborn} % Author's place of birth. Used on the unaccepted title page.
\examdate{15.08.2018} % Day of the oral exam, given as as DD.MM.YYYY. Used on the accepted title page.
\thesisnumber{123} % Thesis number ("EIM-E/" prefix is added automatically). Used on the accepted title page.

\begin{document}
    % The entire frontmatter is automatically generated by the emtthesis class.
    % User-configurable content is found in the following files (found in the "frontmatter" subdirectory):
    % * preface.tex (default for doctoral theses, optional for B.Sc. and M.Sc. theses)
    % * problemstatement.tex (mandatory for B.Sc. and M.Sc. theses)
    % * abstract_de.tex (mandatory)
    % * abstract_en.tex (mandatory for doctoral theses)
    % * symbols.tex (mandatory)

    % \input your regular content chapters (mainmatter).
    \chapter{Einleitung}
Hier erscheint die Einleitung.

    % TODO: Add your own chapters here.
    \chapter{Zusammenfassung und Ausblick}
Hier erscheinen Zusammenfassung und Ausblick.


    % After the regular content, the appendix follows, starting with the bibliography.
    \appendix

    % Insert the bibliography, using the correct single line spacing page layout.
    \insertbibliography%

    % TODO: Add your own appendices here.
    \chapter{Anhang}
Hier erscheint der Anhang.

\end{document}
